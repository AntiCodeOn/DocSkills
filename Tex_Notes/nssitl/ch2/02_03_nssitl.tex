\documentclass{article}
%Titles, Chapters and Sections
\begin{document}
\flushleft
\begin{enumerate}
\item You can nest the list environments to your taste:
\begin{itemize}
\item But it might start to look silly.
\item[-] With a dash.
\end{itemize}
\item Therefore remember:
\begin{description}
\item[Stupid] things will not become smart because they are in a list.
\item[Smart] thins, though, can be presented beautifully in a list.
\end{description}
\end{enumerate}
\newpage
\begin{flushleft}
This text is\\ left aligned. \LaTeX{} is not trying to make each line the same length.
\end{flushleft}
\begin{flushright}
This text is right-\\aligned. \LaTeX{} is not trying to make each line the same length.
\end{flushright}
\begin{center}
At the centre\\ of the earth
\end{center}
\newpage
\begin{tabular}{|r|l|}
\hline
7C0 & hexadecimal \\
3700 & octal \\ \cline{2-2}
111110000000 & binary \\
\hline \hline
1984 & decimal \\
\hline
\end{tabular}
\newline
\newline
\newline
\begin{tabular}{|p{4.7cm}|}
\hline
Welcome to Boxy's paragraph.
We sincerely hope you'll all enjoy the show.\\
\hline
\end{tabular}
\newline
\newline
\newline
\begin{tabular}{@{} l @{}}
\hline
no leading space\\
\hline
\end{tabular}
\newline
\newline
\newline
\begin{tabular}{c r @{.} l}
Pi expression & \multicolumn{2}{c}{Value} \\
\hline
$\pi$ & 3&1416 \\
$\pi^{\pi}$ & 36&46 \\
$(\pi^{\pi})^{\pi}$ & 80662&7 \\
\end{tabular}
\newline
\newline
\newline
\begin{tabular}{|c|c|}
\hline
\multicolumn{2}{|c|}{Ene} \\
\hline
Mene & Muh! \\
\hline
\end{tabular}
\newline
\newline
\newline
\clearpage
Figure~\ref{white} is an example of Pop-Art.
\begin{figure}[!hbtp]
\makebox[\textwidth]{\framebox[5cm]{\rule{Opt}{5cm}}}
\caption{Five by Five in centimeters.\label{white}}
\end{figure}
\end{document}
