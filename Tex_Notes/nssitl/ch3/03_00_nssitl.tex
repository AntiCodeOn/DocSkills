\documentclass{article}
\usepackage{amsmath}
\usepackage{amssymb} %necessary for \mathbb
%\DeclareMathOperator{\argh}{argh}
%\DeclareMathOperator*{\nut}{Nut}
\begin{document}
Add $a$ squared and $b$ squared to get $c$ squared. Or, using a more mathematical approach:
$a^2 + b^2 = c^2$
\newline
\TeX{} is pronounced as $\tau\epsilon\chi$\\[5pt]
100~m$^{3}$ of water\\[5pt]
This comes from my $\heartsuit$
\newline
Add $a$ squared and $b$ squared to get $c$ squared. Or, using a more mathematical approach:
\begin{equation}
a^2 + b^2 = c^2
\end{equation}
Einstein says
\begin{equation}
E = mc^2 \label{clever}
\end{equation}
He didn't say
\begin{equation}
1 + 1 = 3 \tag{dumb}
\end{equation}
This is a reference to \eqref{clever}.
\newline
$\forall x \in \mathbf{R}:
 \qquad x^{2} \geq 0$
\newline
$x^{2} \geq 0\qquad
 \text{for all}x\in\mathbf{R}$
\newline
$x^{2} \geq 0\qquad
 \text{for all}x\in\mathbb{R}$
$\underbrace{\overbrace{a+b+c}^6\cdot\overbrace{d+e+f}^7}_\text{meaning of life} = 42$
\begin{equation*}
 \lim_{x \rightarrow 0}
 \frac{\sin x}{x}=1
\end{equation*}
%\begin{equation*}
% 3\argh = 2\nut_{x=1}
%\end{equation*}
$a\bmod b \\
 x\equiv a \pmod{b}$
\end{document}
