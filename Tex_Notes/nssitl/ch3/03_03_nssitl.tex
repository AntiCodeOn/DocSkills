\documentclass{article}
\newcommand{\ud}{\,\mathrm{d}}
\usepackage{amsmath}
\usepackage{amsfonts}
\usepackage{amssymb}
\usepackage{amsthm}
\theoremstyle{definition} \newtheorem{law}{Law}
\theoremstyle{plain}      \newtheorem{jury}[law]{Jury}
\theoremstyle{remark}     \newtheorem*{marg}{Margaret}
\begin{document}
\begin{equation*}
 \mathbf{X} = \left(
   \begin{array}{ccc}
     x_1 & x_2 & \ldots \\
     x_3 & x_4 & \ldots \\
     \vdots & \vdots & \ddots
   \end{array} \right)
\end{equation*}
\begin{equation*}
  |x| = \left\{
    \begin{array}{rl}
     -x & \text{if } x < 0, \\
     0 &  \text{if } x = 0, \\
     x &  \text{if } x > 0.
    \end{array} \right.
\end{equation*}
\begin{equation*}
 \int_a^b f(x)\ud x
\end{equation*}
\begin{equation*}
{}^{14}_{6}\text{C}
\qquad \text{versus} \qquad
{}^{14}_{\phantom{1}6}\text{C}
\end{equation*}
$\mu, M \qquad \boldsymbol{\mu}, \boldsymbol{M}$
\begin{law} \label{law:box}
Don't hide in the witness box
\end{law}
\begin{jury}[The Twelve]
It could be you! So beware and see law~\ref{law:box}.
\end{jury}
\begin{jury}
You will disregard the last statetment. 
\end{jury}
\begin{marg}No, No, No\end{marg}
\begin{marg}Denis!\end{marg}
\end{document}
